В современном мире существует множество операционных систем и платформ: Linux, Windows, MacOS, Android, iOS, веб-браузеры и другие.
Многие из них тесно связаны друг с другом, однако, в общем случае, приложение, предназначенное лишь для одной платформы, невозможно или проблематично запустить на другой.
Приложение, предназначенное пля работы на нескольких платформах, называют кроссплатформенным.
Для создания кроссплатформенного приложения может быть использована кросс-компиляция, специальные виртуальные машины или оба подхода одновременно. 

Кросс-компиляция -- это процесс компиляции программного кода для одной платформы на другой платформе.
Кросс-компиляция позволяет добиться большей производительности и меньшего размеры исполняемого файла, чем при использовании виртуальных машин.
Большинство компилируемых языков программирования поддерживают кросс-компиляцию.
Однако, кросс-компиляция некоторых языков программирования, таких как C и C++, может оказаться проблематичной, так как функционал целевых платформ или используемых компиляторов может сильно отличаться.

Для использования виртуальных машин, таких как JVM, .NET, V8, QuickJS или CPython, необходимо использовать только те языки программирования, которые поддерживают соответствующую виртуальную машину.
Кроме того, многие виртуальные машины используют компиляцию во время выполнения, что не позволяет использовать их на таких платформах, как iOS и веб-браузеры.
Однако, такой подход позволяет распространять приложение в независимом от платформы формате, к примеру, в формате JAR архива для виртуальной машины JVM или в виде исходного кода для виртуальных машин скриптовых языков программирования.

Использование обоих подходов подразумевает разграничение кросс-компилированного кода и кода, выполняемого в виртуальной машине. 
Такой подход требует больших усилий при проектировании приложения, но позволяет разработчику выбрать, для какого кода будут использованы возможности и соблюдены ограничения того или иного подхода.

Целью дипломного проектировании является создание кроссплатформенной среды выполнения кода, объединяющей преимущества существующих подходов к созданию кроссплатформенных приложений.
