% Цели-задачи-метод-результаты

В настоящее время разработка мобильных приложений стала неотъемлемой частью современной технологической среды,
предоставляющей уникальные возможности и решения в различных областях жизни.
С мобильными приложениями мы можем создавать инновационные продукты, улучшать коммуникацию,
повышать эффективность работы и сделать нашу жизнь более удобной и комфортной.

Благодаря мобильным приложениям мы можем осуществлять заказы и покупки онлайн,
получать доступ к информации и услугам в любое время и в любом месте.
Они стали незаменимыми инструментами в сфере электронной коммерции, финансовых услуг,
медицины, образования, развлечений и других отраслях. Мобильные приложения стали платформой для инноваций и новых идей,
способствуя развитию экономики и повышению качества жизни.

Целью дипломного проектирования является готовая база данных и мобильное приложение
для учета заказов юридических лиц.

% В рамках дипломного проектирования объектом является разработка мобильного приложения.
% Объектом дипломного проектирования является система учета заказов для юридических лиц.
Объектом иследования является система учета заказов.
Предметом - оптимизация процесса заказа товаров для юридических лиц
через разработанное мобильное приложение.

Задачами данного дипломного проектирования является обследование объекта автоматизации.

Проектирование базы данных и реализация мобильного приложения ведётся следующими методами:
описываем программу в виде UML диграмы предецентов, диграмы последовательности, диграмы развертывания;
описываем базу данных в виде логической модели в draw.io;
описываем таблицы базы данных в виде класcов на языке программирования TypeScript;
реализуем Rest API приложение на фреймворке NestJS на языке TypeScript;
документируем HTTP запросы с помощью Swagger UI;
реализуем мобильное приложение с помощью библиотеки React Native на языке TypeScript.

Результатом дипломного проектирования является мобильное приложение и готовая база данных,
которая хранит данные о брендах, категориях, номенклатуре,
хакрактеиристиках номенклатуры,
списка картинок номенклатуры,
данных юридических лиц,
данных о заказе и его статусе.

В рамках дипломного проектирования представлен комплексный подход,
включающий:
проектирование базы данных;
разработку серверной части (backend) для обеспечения Rest API приложения;
разработку веб-сайта с панелью администратора (frontend), предназначенной для редактирования брендов, категорий и номенклатуры;
разработку веб-сайта с панелью менеджера (frontend), обеспечивающий просмотр заказов из базы данных;
создание мобильного приложения (frontend),
которое позволяет осуществлять заказ номенклатуры.

В первом разделе
проводим анализ существующих мобильных приложений таких как
WildBerries, OZ.by, LC Waikiki, Lamoda, De Facto, Ali Express;
выбираем базу данных, средства для реализации серверной части (backend) и мобильного приложения (frontend).

Во втором разделе
описываем приложение UML диграмой прецедентов;
описываем работу с серверной частью UML диграммой последовательности;
приводим в виде таблицы роли, HTTP коды и пути серверной части. 
В этом разделе описываем логическую модель и переводим ее в физическую модель используя миграции TypeORM.

В третьем разделе
работа приложения показазываем на UML диаграмме развертывания;
проведены испытания для мобильного приложения, панели администратора и панель менеджера.
% представлена документация SwaggerUI, которая играет важную роль в тестировании эндпоинтов.
% SwaggerUI предоставляет удобный интерфейс для взаимодействия с эндпоинтами системы,
% позволяя разработчикам и тестировщикам отправлять запросы на серверную часть (backend)
% и получать соответствующие ответы.
% Кроме того, в третьем разделе приведены скриншоты мобильного приложения,
% которые позволяют ознакомиться с его внешним видом и пользовательским интерфейсом.
% Эти изображения предоставляют визуальное представление о функциональности и возможностях приложения,
% позволяя получить наглядное представление о его работе.

В четвёртом разделе
производится сравнение объема функций из каталога с уточненным объемом функций разрабатываемого проекта.
Осуществляется расчет себестоимости оборудования по различным статьям,
включая отчисления на социальные нужды, материалы и комплектующие,
машинное время, накладные расходы, а также затраты на освоение и сопровождение программного средства.
Также производится расчет плановой прибыли, прогнозируемой цены без налогов, отпускной цены и
расчета прибыли от реализации программного обеспечения за вычетом налога на прибыль.

% В приложении А
% представлено техническое задание, которое поможет лучше понять требования к проекту.

% В приложении Б
% представлены данные для проверки базы данных, что позволяет убедиться в корректности ее работы.

% В приложении В
% можно ознакомиться с результатами загрузки данных в базу данных.

% В приложении Г
% представлена инструкция по установке необходимого программного обеспечения для успешной настройки и запуска проекта.
% Инструкция включает установку NodeJS, установку пакетного менеджера yarn,
% а также установку Docker для запуска БД на операционной системе Windows 10.
% % Эти инструменты необходимы для работы с серверной частью (backend), разработанной на NestTS,
% % мобильного приложения (frontend), написанного на React Native с использованием TypeScript,
% % а также веб-сайтом с панелью администратора (frontend), разработанным с использованием NextTS.
% % Эта инструкция обеспечит удобство и понятность при настройке необходимого программного обеспечения,
% % что позволит пользователям успешно разрабатывать и демонстрировать проект.

% В приложении Д приведен каталогом API,
% где можно ознакомится со списком картинок с их путями (URL) и HTTP методами офромленными в SwaggerUI.

% В приложении E
% прикреплен диск, содержащий исходные коды серверной части, базы данных,
% мобильного приложения, веб-приложения с панелью администратора и менеджера,
% а также отчет, оформленный в LaTeX.

В приложении А
прикреплен диск, содержащий исходные коды серверной части,
мобильного приложения,
веб-приложения с панелью администратора и с панелью менеджера.
На диске также содержится отчет, оформленный в LaTeX.

\newpage
