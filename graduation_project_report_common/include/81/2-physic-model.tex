\subsection{Физическая модель}

При работе с базами данных в NestJS, может понадобиться изменение её структуры, например, добавление новых таблицы или изменение существующих.
Для этого в NestJS используются миграции.

Миграции - это специальные файлы, которые содержат инструкции для изменения структуры базы данных.
Они позволяют изменять схему базы данных без необходимости вручную изменять ее структуру.

В NestJS миграции генерируются с помощью инструмента командной строки TypeORM.
Для генерации новой миграции необходимо использовать команду <<typeorm migration:generate>>.
В этой команде указаваем имя миграции.

Когда миграция готова, можно применить её к базе данных, используя команду <<typeorm migration:run>>.
Эта команда выполнит все новые миграции и обновит структуру базы данных в соответствии изменениями.

Таким образом, генерация миграций в NestJS с помощью TypeORM позволяет управлять изменениями в базе данных и обеспечивает более безопасную работу с данными.

TypeORM также предоставляет команду <<migration:revert>>, которая позволяет отменять миграции и откатывать изменения, сделанные в базе данных.

Команда <<migration:revert>> используется для отмены последней примененной миграции.
Это означает, что если уже применили миграцию к базе данных с помощью команды <<migration:run>>,
то команда <<migration:revert>> откатит последнее изменение в базе данных, которое было сделано в этой миграции.
