% Кросскомпиляция -- это процесс компиляции программного кода для одной платформы на другой платформе.
% Большинство компилируемых языков программирования поддерживают кросскомпиляцию.
Нативный код - это код, предназначений для выполнения на ЦП.
Нативный код также называют машинным кодом.
Нативный код чаще всего предназначен только для одной платформы, и его выполнение на других платформах невозможно или затруднительно.
Распространение приложений в виде нативного кода обеспечивает высокую производительность и малое время запуска.

Компилятор — это программное обеспечение, которое переводит исходный код в нативный.
Компиляторы, которые запускаются на одной платформе, но создают нативный код, предназначенный для выполнения на другой платформе, называют кросс-компиляторами.
Большая часть основополагающих программ и программных библиотек написаны на языках C и C++, потому для них существует большое количество компиляторов.
Самыми популярными компиляторами являются GCC, Clang и MSVC, а также их варианты.

GCC -- это компилятор от GNU, поддерживающий несколько языков программирования, в том числе C и C++. 
GCC может быть скомпилирован таким образом, чтобы целевая платформа или архитектура процессора, для которых GCC будет компилировать программный код, отличались от тех, на которых работает он сам, что позволяет использовать его в качестве кросс-компилятора.
Наибольшее распространение GCC получил в области компиляции программ для ОС Linux и встраиваемых систем.
MinGW-GCC, компилятор, созданный на основе GCC, позволяет компилировать приложения для ОС Windows.

Clang -- это компилятор от LLVM, поддерживающий языки программирования C и C++.
Библиотека LLVM, которая является ключевым элементом данного компилятора, позволяет легко использовать его качестве кросскомпилятора. 
Clang поддерживает большинство существующих платформ. 
Для некоторых платформ существуют компиляторы, созданные на основе Clang:
\begin{itemize}
    \item[-] Clang-CL для ОС Windows;
    \item[-] Apple-Clang для таких платформ, как MacOS и iOS;
    \item[-] Emscripten для веб-браузеров.
\end{itemize}

MSVC -- это проприетарный компилятор от Microsoft, поддерживающий языки программирования C и C++. 
MSVC предназначен только для ОС Windows. 

Следует отметить, что написание программного кода на языках C и C++, который бы компилировался для различных платформ, может оказаться сложной задачей, так как функционал целевых платформ или используемых компиляторов может сильно отличаться.



% Использование обоих подходов подразумевает разграничение кросскомпилированного кода и кода, выполняемого в виртуальной машине. 
% Такой подход требует больших усилий при проектировании приложения, но позволяет разработчику выбрать, для какого кода будут использованы возможности и соблюдены ограничения того или иного подхода.

% Целью дипломного проектировании является создание кроссплатформенной среды выполнения кода, объединяющей таких преимущества существующих подходов к созданию кроссплатформенных приложений, что позволит упростить разработку и установку кроссплатформенных приложений.
