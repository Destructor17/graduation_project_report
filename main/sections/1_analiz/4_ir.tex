Программный код может быть скомпилирован в нативный код не напрямую, а посредствам еще одного вида кода, называемого промежуточным представлением.
Такой подход позволяет упростить кросскомпиляцию и реализацию новых языков программирования. 
Зачастую промежуточное представление может быть запущено в виртуальной машине.

Примерами промежуточного представления являются LLVM IR, .NET CIL и Java bytecode.

LLVM IR создан в рамках проекта LLVM и значительно упрощает процесс создания обычных и JIT компиляторов. 
Компиляторы, использующие LLVM IR, существуют для многих языков программирования, включая C, C++, Fortran, Rust, Pascal, Haskell и Kotlin. 
Однако, LLVM IR плохо подходит для распространения программ, так как сильно зависит от целевой платформы и архитектуры процессора \cite{LLVMNotPortable}.
Компилляция LLVM IR в машинный код может занять значительное количество времени.

.NET CIL предназначен в первую очередь для последующей JIT компиляции. 
Однако, Native AOT позволяет компилировать .NET CIL в машинный код.
Это позволяет распространять приложения, скомпилированные в .NET CIL, как без изменений, так и посредством преобразования в нативные приложения.
Следует отметить, что Native AOT использует библиотеку LLVM.

Изначально предполагалось, что Java bytecode будет использоваться следующим образом: программный код на языке Java компилируется в Java bytecode, Java bytecode используется для распространения приложений и для интерпретации и JIT компиляции.
Кроме того, что со временем появились новые языки программирования, поддерживающие Java bytecode, появился GraalVM.
GraalVM предназначен для компиляции Java bytecode в нативный код.
В этом случае Java bytecode выступает в качестве промежуточного представления.
GraalVM также использует LLVM.
