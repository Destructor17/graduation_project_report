Этот подход подразумевает разделение программного кода приложения на две части. 
Первая часть выполняется в виртуальной машине.
Вторая часть кода компилируется в нативный код. Как правило во второй части оказывается либо код, сильно зависящий от целевой платформы, либо код, требующий высокой производительности.

Использование нативного кода в купе с виртуальной машиной позволяет частично объединить преимущества двух подходов. 
Однако, налаживание взаимодействия между виртуальной машиной и нативным кодом может оказаться сложной задачей.
