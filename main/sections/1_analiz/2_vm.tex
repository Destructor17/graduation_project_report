Объектный код — это код, который является результатом компиляции исходного кода
В большинстве случаев объектный код не является человекочитаемым.
Объектный код, предназначенный для выполнения в виртуальной машине, называется байт-кодом \cite{BytecodeDefentition}.
Виртуальные машины позволяют выполнять код программ с минимальной зависимостью от платформы и архитектуры процессора.
Таким образом, в большинстве случаев достаточно единожды скомпилировать исходный код в байт-код, который может быть запущен на любой платформе.

Некоторые виртуальные машины не требуют предварительной компиляции в байт-код, что позволяет распространять приложения в виде исходного кода.
Языки программирования для таких виртуальных машин называют скриптовыми.

Существует большое количество виртуальных машин и их реализаций.
Яркими примерами являются JVM, .NET, а также виртуальные машины скриптовых языков программирования, таких как JavaScript и Python.

Самым большим и распространенным недостатком использования виртуальных машин является ограничение в выборе языков программирования. 
Например, JVM поддерживается лишь языками Java, Scala, Kotlin и некоторыми другими. 
Виртуальные машины скриптовых языков, как правило, поддерживают лишь один язык программирования.

В случае, если на целевом устройстве уже установлена реализация виртуальной машины, то проявляется еще одно отличие данного подхода - портативность. 
К примеру, приложение может быть собрано в формате JAR лишь единожды, но запускать на различных платформах, на которых установлена реализация виртуальной, не требуя каких либо изменений или дополнительных шагов во время установки.

Для выполнения кода виртуальные машины используют используют интерпретацию, JIT-компиляцию или оба подхода одновременно.

Интерпретация подразумевает выполнение кода виртуальной машиной без преобразования в нативный кода. 
В сравнении с JIT-компиляторами, интерпретаторы намного проще в реализации, мало зависят от архитектуры процессора, позволяет быстро запустить приложение.
Но интерпретаторы выполняют дополнительную работу по анализу и интерпретации кода во время выполнения, что значительно понижает производительность.

JIT-компиляция подразумевает, что во время работы виртуальной машины код компилируется в нативный и результат записывается в исполняемую область памяти для дальнейшего выполнения.
За счёт выполнения нативного кода напрямую достигается высокая производительность.
Но JIT-компиляторы сложны в реализации и сильно зависят от архитектуры процессора.
Кроме того, может потребоваться значительное количество времени для компиляции, зто может замедлить запуск приложения.

Некоторые реализации виртуальных машин, такие как Oracle HotSpot и V8, используют как интерпретацию, так и JIT-компиляцию. 
Изначально код интерпретируется, но если участок кода выполняется достаточно часто, то он JIT-компилируется для повышения производительности. 
Такой подход позволяет совместить скорость запуска интерпретаторов и скорость выполнения JIT-компиляторов.

Следует отметить, что на некоторых платформах, таких как iOS, существуют ограничения, делающие выделение исполняемой памяти невозможным, что не позволяет на них использовать JIT-компилятор. \cite{IOSJIT}

Программный код приложения также может быть разделен на две части, где первая часть выполняется в виртуальной машине, тогда как вторая часть компилируется в нативный код. 
Как правило во второй части оказывается либо код, сильно зависящий от целевой платформы, либо код, требующий высокой производительности.
Использование нативного кода в купе с виртуальной машиной позволяет частично объединить преимущества двух подходов. 
Однако, налаживание взаимодействия между виртуальной машиной и нативным кодом может оказаться сложной задачей.
