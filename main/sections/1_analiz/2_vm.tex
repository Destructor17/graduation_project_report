Виртуальные машины позволяют выполнять код программ с минимальной зависимостью от платформы и архитектуры процессора.

Существует большое количество виртуальных машин и их реализаций.
Яркими примерами являются JVM, .NET, а также виртуальные машины скриптовых языков программирования, таких как JavaScript и Python.

Самым большим недостатком такого подхода является ограничение в выборе языков программирования. 
Например, JVM поддерживается лишь языками Java, Scala, Kotlin и некоторыми другими. 
Виртуальные машины скриптовых языков, как правило, поддерживают лишь один язык программирования.

В случае, если на целевом устройстве уже установлена реализация виртуальной машины, то проявляется еще одно отличие данного подхода - портативность. 
К примеру, приложение может быть собрано в формате JAR лишь единожды, но запускать на различных платформах, не требуя каких либо изменений.

Для выполнения кода виртуальные машины используют используют интерпретацию, JIT-компиляцию или оба подхода одновременно.

Интерпретация подразумевает пошаговое выполнение кода самой виртуальной машиной. Интерпретаторы намного проще в реализации, мало зависят от архитектуры процессора, позволяет быстро запустить приложение, но такой подход сильно уменьшает производительность.

JIT-компиляция подразумевает генерацию исполняемого машинного кода и запись его в исполняемую область памяти для дальнейшего выполнения.
JIT-компиляторы сложны в реализации, сильно зависят от архитектуры процессора, могут требовать значительного количества времени для компиляции, но обеспечивают высокую производительность за счет выполнения машинного кода напрямую.

Некоторые реализации виртуальных машин, такие как Oracle HotSpot и V8, используют оба  ранее перечисленных подхода. 
Изначально код интерпретируется, но если участок кода выполняется достаточно часто, то он JIT-компилируется для повышения производительности. 
Такой подход позволяет совместить скорость запуска интерпретаторов и скорость выполнения JIT-компиляторов.

Следует отметить, что на некоторых платформах, таких как iOS, существуют ограничения, делающие выделение исполняемой памяти невозможным, что не позволяет на них использовать JIT-компилятор. \cite{IOSJIT}
