Кросскомпиляция -- это процесс компиляции программного кода для одной платформы на другой платформе.
Большинство компилируемых языков программирования поддерживают кросскомпиляцию.
Однако, кросскомпиляция некоторых языков программирования, таких как C и C++, может оказаться проблематичной, так как функционал целевых платформ или используемых компиляторов может сильно отличаться.


% Использование обоих подходов подразумевает разграничение кросскомпилированного кода и кода, выполняемого в виртуальной машине. 
% Такой подход требует больших усилий при проектировании приложения, но позволяет разработчику выбрать, для какого кода будут использованы возможности и соблюдены ограничения того или иного подхода.

% Целью дипломного проектировании является создание кроссплатформенной среды выполнения кода, объединяющей таких преимущества существующих подходов к созданию кроссплатформенных приложений, что позволит упростить разработку и установку кроссплатформенных приложений.
