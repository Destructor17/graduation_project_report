В файле make\_webrogue.cmake и в директории cmake/ находится код, используемый в других файлах конфигурации для различных платформ.

В файле CMakeLists.txt находятся конфигурации сборки для ОС Windows, Linux и MacOS, предназначенные для тестирования и отладки среды выполнения.

В директории platforms/ находятся отдельные директории для поддерживаемых платформ. Данные конфигурации предназначены в первую очередь для релизных сборок.

В файле platforms/Windows/CMakeLists.txt находятся конфигурации сборки для ОС Windows. 
Эта конфигурация работает со средой сборки MSBuild и компиллятором MSVC, что позволяет использовать её на устройствах под управлением ОС Windows и установленным программным обеспечением Microsoft Visual Studio.
Эта конфигурация поддерживает сборку установщика в виде исполняемого файла с помощью Nullsoft Scriptable Install System, а также в формате MSIX.
Для работы с этой конфигурацией создан вспомогательный скрипт platforms/Windows/build.ps1.

В файле platforms/Linux/CMakeLists.txt находятся конфигурации сборки для ОС Linux. 
Данная конфигурация поддерживает компилляторы GCC и CLang, а также системы сборки GNU Make и Ninja.
Эта конфигурация поддерживает сборку пакетов DEB и RPM.
Сборка для ОС Linux может быть произведена без использования ПО Docker, однако скрипт platforms/Linux/ubuntu.sh позволяют произвести сборку на любом устройстве с установленным ПО Docker.
Скрипт platforms/Linux/deb2appimage.sh позволяют собрать приложение в формате AppImage, который значительно облегчает процесс установки собраного приложения.

В файле platforms/MacOS/CMakeLists.txt находятся конфигурации сборки для MacOS. 
Эта конфигурация работает со средой сборки XCBuild и компиллятором Apple-Clang.
Данная конфигурация может быть использована на устройствах под управлением MacOS и установленным программным обеспечением XCode.

В директории platforms/Android/ находится проект, совместимый с Android Studio.
Большая часть этого проекта написана на языке программирования Java, но также включает файл конфигурации platforms/Android/app/jni/CMakeLists.txt.
Так как Android SDK имеет встроенную поддержку Cmake, сборка Android-приложения

В файле platforms/iOS/CMakeLists.txt находятся конфигурации сборки для iOS. 
В директории platforms/iOS/ также находится код на языках программирования Swift и Objective-C++.

В директории platforms/Web/ находятся реализации среды выполнения кода для веб-браузеров, которые могут быть скомпилированы с использованием Emscripten.
Скрипт scripts/webtest.sh компилирует код, находящийся в директории platforms/Web/, и запускает простой веб-сервер в целях отладки и демонстрации.
