Wasm2c -- это программа, преобразующая код WebAssembly в код на языке программирования C.
При этом код WebAssembly может скомпилирован из программного кода других языков программирования, к примеру C++.

Запекание Wasm2c подразумевает компиляцию программного кода, полученного с помощью программы Wasm2c, в машинный код на этапе сборки среды выполнения кода.
В этом случае код WebAssembly выступает в роли промежуточного представления.
Такой подход, как и метод нативного запекания, обеспечивает высокую производительности и скорость запуска приложения, но не позволяет запустить любое приложение, кроме того, для которого среда была скомпилирована, и не позволяет добавлять плагины после компиляции.

Главным преимуществом данного подхода перед методом нативного запекания является простота компиляции.
При использовании метода запекания Wasm2c разработчику не нужно волноваться о том, чтобы его программный код мог быть скомпилирован под все поддерживаемые платформы.
К примеру, если разработчик, использующий ОС Linux, создаёт приложение для разрабатываемой среды выполнения кода, он может быть уверен, что с помощью метода запекания Wasm2c это приложение может быть собрано для других ОС Windows или Android.
