WASI позволяет приложений с кодом WebAssembly использовать такие функции операционной системы, как доступ к файловой системе.
WASI представлен в виде набора импортируемых функций.
Самой популярной реализацией WASI является программная библиотека UVWASI, использующая LibUV.
LibUV является кроссплатформенной программной библиотекой для доступа к функциям операционной системы.
LibUV создаёт ограничения к версиям операционной системы. 
К примеру, для ОС Android требуется версия Android 7.0 или выше.
Кроме того, при использовании компилятора Emscripten, для некоторых функций LibUV отсутствуют реализации.
Однако, так как эти функции не используются разрабатываемой средой выполнения кода, для них можно создать пустые реализации.
Данные реализации определены в файле platforms/Web/uv\_emscripten.c.
Так как при использовании метода нативного запекания программный код компилируется с помощью сторонних компиляторов, стандартные библиотеки которых не используют WASI, компиляция UVWASI и LibUV в данном случае не требуется.
