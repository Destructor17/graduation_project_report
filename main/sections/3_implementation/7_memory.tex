В большинстве случаев память приложения с кодом WebAssembly представляет собой один непрерывный участок памяти, также называемый линейной памятью.
В этой памяти располагаются стек, куча, глобальные переменные и некоторые константы.

!!!Image there!!!



Тогда как многие библиотеки для выполнения кода WebAssembly предоставляют функции, позволяющие получить адрес начала линейной памяти и её размер, реализовать такой подход в среде выполнения кода, использующей WebAssembly API, невозможно, так как память основной части среды выполнения кода и память приложения находятся в двух разных объектах ArrayBuffer, один из которых находится в изолированном Web Worker. 
Для реализации доступа к памяти приложения из среды выполнения кода используется подход, подобный подходу, использованному для вызова импортируемых функций из WebAssembly API.
Среда выполнения использует Atomics для уведомления Web Worker о том, что требуется произвести манипуляцию с памятью приложения, после чего Web Worker отправляет сообщение с объектом SharedArrayBuffer, содержащим запрошенным участком памяти, с которым среда выполнения кода может производить такие манипуляции, как чтение и запись данных.

