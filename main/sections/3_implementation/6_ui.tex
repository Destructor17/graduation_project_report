Как сказано ранее, пользовательский интерфейс представляется в виде матрицы, элементы которой содержат информацию о символе, его цвете и цвете фона.
Самым простым способом отобразить такую матрице является программная библиотека NCurses.

NCurses позволяет с текстовыми пользовательскими интерфейсами на таких операционных системах, как Linux и MacOS.
Эмулятор терминала Windows сильно отличается от таковых у Linux и MacOS, потому NCurses их не поддерживает.
Для ОС Windows существует программная библиотека PDCurses, предоставляющая функционал, подобный таковому у NCurses.
