Как сказано ранее, пользовательский интерфейс представляется в виде матрицы, элементы которой содержат информацию о символе, его цвете и цвете фона.
Самым простым способом отобразить такую матрице является программная библиотека NCurses.

NCurses позволяет работать с текстовыми пользовательскими интерфейсами на таких операционных системах, как Linux и MacOS.
Эмулятор терминала Windows сильно отличается от таковых у Linux и MacOS, потому NCurses их не поддерживает.
Для ОС Windows существует программная библиотека PDCurses, предоставляющая функционал, подобный таковому у NCurses.

На многих платформах, таких как Android, iOS и веб-браузеры, нет встроенного эмулятора терминала или его использование затруднительно.
Для таких платформ использование библиотек NCurses и PDCurses невозможно, поэтому для них пользовательский интерфейс реализован с помощью SDL.

Программная библиотека SDL предоставляет возможность простых операций с графикой, таких как отрисовка прямоугольников и копирование частей изображения.
Программная библиотека SDL\_ttf также предоставляет возможность отрисовки текста.
SDL и SDL\_ttf поддерживаются большинством существующих платформ.
На платформах Android, iOS и веб-браузерах для реализации пользовательского интерфейса используется только SDL и SDL\_ttf.
На других платформах существует возможность выбора. 
К примеру, для ОС Windows разрабатываемая среда выполнения кода предоставляет два исполняемых файла: webrogue.exe и webrogue\_sdl.exe.
webrogue.exe использует PDCurses, и открывает встроенный эмулятор терминала Windows, тогда как webrogue\_sdl.exe открывает новое окно, в котором пользовательский интерфейс реализован с помощью SDL.
