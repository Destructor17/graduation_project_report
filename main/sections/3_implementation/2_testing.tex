Для демонстрации проверки системы рассмотрим последовательности следующих тестов:
\begin{itemize}
    \item[] Тест 1 -- Проверка работы на ОС Linux с графическим интерфейсом.
    \item[] Тест 2 -- Проверка работы на ОС Linux без графического интерфейса.
    \item[] Тест 3 -- Проверка работы на ОС Windows.
    \item[] Тест 4 -- Проверка работы на MacOS.
    \item[] Тест 5 -- Проверка работы на Android.
    \item[] Тест 6 -- Проверка работы на iOS.
    \item[] Тест 7 -- Проверка работы запеченного приложения в веб-браузере с поддержкой WebAssembly.
    \item[] Тест 8 -- Проверка работы запеченного приложения в веб-браузере без поддержки WebAssembly.
    \item[] Тест 9 -- Проверка работы динамической загрузки приложений в веб-браузере.
    \item[] Тест 10 -- Проверка работы приложения rugpt2.
    \item[] Тест 11 -- Проверка работы плагинов.
\end{itemize}

\newcommand{\TestCase}[6]{{
    Тест #1. #2.
    
    \textit{Описание состояния}. #3
    
    \textit{Описание проверки}. #4
    
    \textit{Процедура реализации}. #5

    \textit{Оценка результатов проверки}. #6
}}

\TestCase{1}{Проверка работы на ОС Linux}{
    Среда выполнения кода загружена в формате AppImage с веб-сайта на персональный компьютер под управлением Kubuntu 23.10.
    Дополнительные шаги для установки не требуются.
}{
    Проверяется процесс процесс работы среды выполнения кода для ОС Linux, а также графические возможности среды выполнения кода.
}{
    Для запуска среды выполнения кода достаточно выбрать её с помощью файлового менеджера и, при необходимости, подтвердить добавление прав на выполнение \GPRSeeImageRef{test_linux_permission}.
    Среда выполнения кода также может быть запущена из коммандной строки.

    \GPRImage{test_linux_permission}{Окно подтверждения добавления прав на выполнение на ОС Linux}{main/sections/3_implementation/images/linux_permission.png}{16cm}
}{
    При запуске среды выполнения кода запускается приложение log2048, которое распространяется вместе с самой средой выполнения кода.
    Также открывается окно с графическим интерфейсрм приложения \GPRSeeImageRef{test_linux_sdl}.
    
    \GPRImage{test_linux_sdl}{Окно с графическим интерфейсом на ОС Linux}{main/sections/3_implementation/images/linux_sdl.png}{16cm}
}

\TestCase{2}{Проверка работы на ОС Linux без графического интерфейса}{
    Среда выполнения кода загружена в формате AppImage с веб-сайта на персональный компьютер под управлением Kubuntu 23.10.
    Дополнительные шаги для установки не требуются.
}{
    Проверяется процесс процесс работы среды выполнения кода для ОС Linux, а также возможности текстового интерфейса.
}{
    Для того, чтобы отключить возможность доступа к графическому интерфейса следует удалить переменные окружения DISPLAY и WAYLAND\_DISPLAY.
    Среда выполнения кода может быть запущена из большенстав эмуляторов терминала, в встоенного в Visual Studio Code.
}{
    При запуске среды выполнения кода запускается приложение log2048 и его интерфейс в эмуляторе терминала \GPRSeeImageRef{test_linux_ncurses}. 

    \GPRImage{test_linux_ncurses}{Текстовый интерфейс на ОС Linux}{main/sections/3_implementation/images/linux_ncurses.png}{16cm}
}

\TestCase{3}{Проверка работы на ОС Windows}{
    Установщик среды выполнения кода загружен с веб-сайта и установлен на виртуальную машину под управлением Windows 11.
    Для установки необходимо запустить установщик, разрешить выполнение от имени администратора и следовать появляющимся на экране инструкциям \GPRSeeImageRef{test_windows_installer}

    \GPRImage{test_windows_installer}{Окно мастера установки для ОС Windows}{main/sections/3_implementation/images/windows_installer.png}{14cm}
}{
    Проверяется процесс процесс работы среды выполнения кода для ОС Windows.
}{
    Для запуска среды выполнения кода в графическом режиме достаточно запустить файл webrogue\_sdl.exe, который находится в папке, в которую установлена среда выполнения кода.
}{
    При запуске среды выполнения кода запускается приложение log2048 и графический интерфейс \GPRSeeImageRef{test_windows_sdl}. 

    \GPRImage{test_windows_sdl}{Графический интерфейс на ОС Windows}{main/sections/3_implementation/images/windows_sdl.png}{14cm}
}

\TestCase{4}{Проверка работы на MacOS}{
    Среда выполнения кода загружен с веб-сайта и установлена на устройство под управлением MacOS.
    Для установки необходимо запустить загруженый файл, после чего переместить среду выполнения кода к остальным приложениям \GPRSeeImageRef{test_macos_installer}.

    \GPRImage{test_macos_installer}{Окно установки для MacOS}{main/sections/3_implementation/images/macos_installer.png}{14cm}
}{
    Проверяется процесс процесс работы среды выполнения кода для MacOS.
}{
    Для запуска среды выполнения кода в графическом режиме достаточно запустить соответствующее приложение из меню.
}{
    При запуске среды выполнения кода запускается приложение log2048 и графический интерфейс \GPRSeeImageRef{test_macos_sdl}. 

    \GPRImage{test_macos_sdl}{Графический интерфейс на MacOS}{main/sections/3_implementation/images/macos_sdl.png}{14cm}
}

\TestCase{5}{Проверка работы на Android}{
    Среда выполнения кода загружен с веб-сайта и установлена на устройство  под управлением Android.
    Для установки системный установщик APK-файлов.
}{
    Проверяется процесс процесс работы среды выполнения кода для Android.
}{
    Для запуска среды выполнения кода в графическом режиме нужно запустить соответствующее приложение из меню или на домашнем экране.
    Для запуска приложения нужно нажать на кнопку <<RUN>> \GPRSeeImageRef{test_android_sdl}.
}{
    При запуске среды выполнения кода открывается меню с выбором приложений.
    Приложение log2048 предустановлено.
    При нажатии на кнопку <<RUN>> запускается приложение log2048 и графический интерфейс \GPRSeeImageRef{test_android_sdl}. 

    \GPRImage{test_android_sdl}{Меню выбора приложений и графический интерфейс на Android}{main/sections/3_implementation/images/android_sdl.png}{14cm}
}

\TestCase{6}{Проверка работы на iOS}{
    Среда выполнения кода собрана с помощью XCode и запущена на MacOS с использованием iOS-симулятора.
}{
    Проверяется процесс процесс работы среды выполнения кода для iOS.
}{
    Запуск среды выполнения кода производится с помощью XCode или меню приложений iOS-симулятора.
    Для запуска приложения нужно нажать на треугольную кнопку \GPRSeeImageRef{test_ios_sdl}.
}{
    При запуске среды выполнения кода открывается меню с выбором приложений.
    Приложение log2048 предустановлено.
    При нажатии на кнопку <<RUN>> запускается приложение log2048 и графический интерфейс \GPRSeeImageRef{test_ios_sdl}. 

    \GPRImage{test_ios_sdl}{Меню выбора приложений и графический интерфейс на iOS}{main/sections/3_implementation/images/ios_sdl.png}{14cm}
}

\TestCase{7}{Проверка работы запеченного приложения в веб-браузере с поддержкой WebAssembly}{
    Приложения находится на веб-сайте среды выполнения по адресу https://webrogue-runtime.github.io/demos/baked\_log2048/ и запускается в веб-браузере Firefox.
}{
    Проверяется процесс процесс работы запеченного приложения в современных веб-браузерах.
}{
    Запуск приложения происходит при переходе по адресу https://webrogue-runtime.github.io/demos/baked\_log2048/
}{
    При переходе по указаному адресу запускается приложение log2048 и на веб-странице появляется графический интерфейс \GPRSeeImageRef{test_web_wasm_sdl}. 
    При запуске среды выполнения кода открывается меню с выбором приложений.
    Приложение log2048 предустановлено.
    
    \GPRImage{test_web_wasm_sdl}{Веб-страница со встроеным графическим интерфейсом}{main/sections/3_implementation/images/web_wasm_sdl.png}{9cm}
}

\TestCase{8}{Проверка работы запеченного приложения в веб-браузере без поддержки WebAssembly}{
    Приложения находится на веб-сайте среды выполнения по адресу https://webrogue-runtime.github.io/demos/baked\_log2048/ и запускается в веб-браузере Firefox.
    Поддежка WebAssembly отключается с помощью перехода по адресу about:config в выставления параметра javascript.options.wasm на значение false.
}{
    Проверяется процесс процесс работы запеченного приложения в веб-браузерах без поддержки WebAssembly.
}{
    Запуск приложения происходит при переходе по адресу https://webrogue-runtime.github.io/demos/baked\_log2048/
}{
    При переходе по указаному адресу запускается приложение log2048 и на веб-странице появляется графический интерфейс, идентичный таковому в случае поддержки WebAssembly \GPRSeeImageRef{test_web_wasm_sdl}. 
    При запуске среды выполнения кода открывается меню с выбором приложений.
    Приложение log2048 предустановлено.
}

\TestCase{9}{Проверка работы динамической загрузки приложений в веб-браузере}{
    Приложения находится на веб-сайте среды выполнения по адресу https://webrogue-runtime.github.io/demos/dynamic/ и запускается в веб-браузере Firefox.
}{
    Проверяется процесс процесс работы запеченного приложения в современных веб-браузерах.
}{
    Запуск приложения происходит с при переходе по адресу https://webrogue-runtime.github.io/demos/baked\_log2048/
}{
    При переходе по указаному адресу запускается приложение log2048 и на веб-странице появляется графический интерфейс, идентичный таковому в случае поддержки WebAssembly \GPRSeeImageRef{test_web_dynamic_sdl}. 
    При запуске среды выполнения кода открывается меню с выбором приложений.
    Приложение log2048 предустановлено.

    \GPRImage{test_web_dynamic_sdl}{Веб-страница со встроеным графическим интерфейсом}{main/sections/3_implementation/images/web_dynamic_sdl.png}{9cm}
}
