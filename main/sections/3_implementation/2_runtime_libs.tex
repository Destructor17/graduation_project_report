Существует множество программных библиотек, предоставляющих возможность выполнить код WebAssembly с помощью интерпретации или JIT-компиляции.
Одним из самых популярных решений является Wasmer.

Wasmer написан на языке программирования Rust и предоставляет возможность выполнять код WebAssembly с помощью JIT-компилляции.
Для JIT-компиляции может быть использована одна из трёх библиотек: LLVM, Cranelift и Singlepass.
Так как Singlepass не обеспечивает высокой производительности, а LLVM требует большого количества времени для компилляции \cite{WasmerRuntimes} и значительно увеличивает размер приложения, Cranelift является оптимальным решением.
В рамках дипломного проекта Wasmer и Cranelift используются на ОС Windows, Linux и Android.

Так как многие другие программные библиотеки для JIT-компиляции, включая Cranelift, не поддерживает 32-х битные архитектуры процессоров, разрабатываемая среда выполнения кода должна также поддерживать интерпретаторы.
Одним из самых производительных интерпретаторов предоставляет программная библиотека WAMR.
Тогда как WAMR предоставляет несколько конфигураций, нас интересует конфигурация Fast Interpreter.
Эта конфигурация используется на 32-х битных устройствах под управлением Android, тогда как в 64-х битных используется Wasmer и Cranelift.

MacOS и iOS имеют предустановленную программную библиотеку JavaScriptCore.
JavaScriptCore предназначен в первую очередь для выполнения кода на языке программирования JavaScript, но, так как JavaScriptCore используется в веб-браузере Safari, он также поддерживает выполнение кода WebAssembly.
JavaScriptCore использует интерпретацию и JIT-компиляцию одновременно, что обеспечивает хорошую производительность и скорость запуска.
Так как JavaScriptCore уже предустановленно, его использование уменьшает размер приложения и сложность сборки.
Wasmer может использовать JavaScriptCore вместо программных библиотек для JIT-компилляции.
Всё это делает использование Wasmer и JavaScriptCore разумным решением в рамках операционных систем MacOS и iOS.

Так как в некоторых версиях iOS JavaScriptCore не поддерживает WebAssembly \cite{JSCNoWasm}, следует добавить проверку поддержки WebAssembly в начале работы приложения для iOS.
Если WebAssembly не поддерживается, следует использовать другой подход к выполнению кода.
Так как iOS не позволяет использовать JIT-компилляцию, следует использовать интерпретатор.
WAMR имеет плохую поддержку iOS, поэтому в создаваемой среде выполнения кода будет использована программная библиотека WASM3.
WASM3 прост в компиляции и предоставляет производительный интерпретатор кода WebAssembly.
