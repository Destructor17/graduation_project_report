В Беларуси последовательно проводится государственная политика в области энергосбережения, что является источником гордости для страны в сфере энергетики. Для дальнейшего снижения энергоемкости ВВП, экономии топливно-энергетических ресурсов и увеличения использования местных видов топлива требуется усиленная работа и значительная инвестиционная поддержка.
Один из эффективных способов сократить негативное воздействие человечества на окружающую среду – повышение энергоэффективности. Современная энергетика, основанная на использовании ископаемых видов топлива, оказывает значительное воздействие на экологический баланс планеты, начиная от добычи и переработки ресурсов до их сжигания для производства тепла и электроэнергии. Кроме того, ископаемая энергетика является ответственной за проблему изменения климата и увеличение концентрации парниковых газов. Таким образом, повышение энергоэффективности становится насущной задачей для всех стран.
Рациональное использование ресурсов органического топлива, повышение эффективности энергопотребления во всех секторах экономики, развитие возобновляемых источников энергии – все это вместе позволит удовлетворить потребности человечества в энергии и способствовать его устойчивому развитию в масштабах планеты.
В Беларуси государственная политика в сфере энергосбережения основывается на приоритетных направлениях повышения эффективности использования энергоресурсов. Технические меры в области энергосбережения реализуются через инвестиции в энергоэффективные проекты в рамках отраслевых и региональных программ, а также на республиканском уровне. Кроме того, в Беларуси завершено строительство атомной электростанции, что приведет к увеличению свободной электроэнергии и переходу от твердых видов топлива к ядерному. Альтернативные источники энергии все еще не могут полностью заменить традиционные, но их мощности растут с развитием технологий и активными исследованиями по повышению энергоэффективности. Сокращение потребления энергии стимулирует переход на альтернативные источники.
Производители современной электроники активно работают над снижением энергопотребления компьютеров, и в этом они достигают значительных успехов. Сравнивая компьютеры десятилетней давности с современными, видно значительное улучшение. Современные компьютеры не только потребляют меньше энергии, но и имеют разнообразные режимы работы.
Режим сна позволяет отключить жесткие диски, сохраняя приложения в оперативной памяти, и позволяет быстро возобновить работу. Потребление в этом режиме составляет 7-10\% от общей мощности системы.
Режим гибернации полностью выключает компьютер, сохраняя данные в отдельный файл. Работа возобновляется медленнее, чем после сна, и потребление составляет 5-10 Ватт.
Полное выключение означает полный выход из системы, с потерей всех несохраненных данных. Работа начинается с новой загрузки системы, и потребление составляет 4-5 Ватт.
Потребление зависит от конфигурации системы и от того, какие задачи выполняются.
Следовательно, снижение энергопотребления компьютеров-серверов можно достичь, переводя их в режим сна и выключая монитор, поскольку он нужен только в редких случаях для управления сервером. Уменьшение потребления электроэнергии мобильными устройствами возможно путем включения режима энергосбережения и закрытия фоновых приложений, которые часто обновляются.
Для снижения потребления электроэнергии клиентскими компьютерами можно использовать энергоэффективные модели, уменьшать яркость монитора и переводить компьютер в режим сна при отсутствии использования на короткое время.
