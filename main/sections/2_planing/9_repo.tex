Для упрощения разработки и хранения кода в рамках дипломного проектирования используется Git.
Git -- это популярная система управления версиями исходного кода.

В качестве сервиса для удаленного хранения Git-репозитория в рамках данного дипломного проектирования используется GitHub, так как он является бесплатным, а также предоставляет такие сервисы, как GitHub Actions и GitHub Pages.

GitHub Actions -- это CI/CD платформа, позволяющая автоматизировать сборку программного обеспечения в облаке GitHub.
В рамках дипломного проектирования GitHub Actions сконфигурирован таким образом, чтобы автоматически компилировать код для всех поддерживаемых платформ когда разработчик публикует изменения в GitHub.
Это позволяет быстро обнаружить изменения, которые не могут быть скомпилированы на некоторых платформах, что значительно упрощает разработку кроссплатформенных приложений с использованием языка C++.

GitHub Pages -- это хостинг статичных веб-сайтов от GitHub.
Статичный веб-сайт -- это сайт, который не использует динамические базы данных или серверные скрипты. 
Вместо этого он состоит из статических файлов, таких как HTML, CSS и JavaScript и WebAssembly.
Отсутствие бэкэнд-систем и запросов к базам данных делает статичные сайты очень простыми, быстрыми и безопасными.

В рамках дипломного проектирования на хостинге GitHub Pages создан веб-сайт, доступный по адресу webrogue-runtime.github.io.
Целями этого сайта является демонстрация работы разрабатываемой среды выполнения кода, а также предоставление упрощение загрузки и установки программного обеспечения.
Для создания веб-сайта использован Jekyll -- простой в использовании генератор статичных веб-сайтов.

На сайте представлено два примера использования разрабатываемой среды выполнения кода.
Оба примера скомпилированы с использованием Emscripten.
Emscripten может компилировать код как в WebAssembly, так и в Asm.js.
Asm.js -- это вид программного кода, который может быть запущен в средах выполнения кода, поддерживающий JavaScript, таких как веб-браузеры.
WebAssembly, Asm.js обеспечивает большую совместимость, но меньшую производительность.

В первом примере использован метод нативного запекания для приложения log2048.
Этот пример представлен в виде двух вариантов: в виде кода WebAssembly и кода Asm.js.
Вариант определяется автоматически в зависимости от того, поддерживает ли веб-браузер выполнение кода WebAssembly.

Второй вариант использует WebAssembly API для выполнения кода приложений.
Так как этот вариант требует поддержки WebAssembly API, он представлен только в виде кода WebAssembly, а потому не может быть запущен на некоторых устаревших браузерах.
Пользователь может загрузить файл приложения для дальнейшего выполнения или удалить уже загруженный.
При нажатии на кнопку запуска начнется исполнение ранее закруженного приложения или. 
Если же приложение не было загружено, в целях демонстрации запустится log2048.

На созданном веб-сайте имеются ссылки для загрузки реализаций разработанной среды выполнения кода для различных платформ.

Созданная в рамках дипломного проектирования конфигурация GitHub Actions имеет команду, которая запускается не автоматически, а вручную.
При запуске этой команды происходит сборка реализаций разработанной среды выполнения для различных платформ, сборка разработанного статичного веб-сайта и развёртывание всего собранного программного обеспечения на GitHub Pages.
