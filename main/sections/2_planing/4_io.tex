Работа приложения подразумевает получение входных и выходных данных в том или ином виде.
WebAssembly предоставляет возможность среде выполнения определить функции, которые могут быть вызваны из приложения, работающего в этой среде.
Такие функции называются импортируемыми.
Импортируемые функции могут быть использованы для реализаций функций ввода и вывода данных.

Большинство приложений предоставляют пользовательский интерфейс, что делает их более интерактивными и простыми в использовании.
Сложность существующих программных библиотек для пользовательского интерфейса делает реализацию соответствующих импортируемых функций трудоёмкой.
В рамках данного дипломного проекта целесообразно использовать упрощённый подход: пользовательский интерфейс представляется в виде матрицы, элементы которой содержат информацию о символе, его цвете и цвете фона.
Для представления символа целесообразно использовать кодировку UTF-32, так как она поддерживает все существующие символы, и представление каждого символа требует одинакового количества байт.

В большинстве случаев приложения также манипулируют файлами.
WASI предоставляет набор стандартизированных импортируемых функций для взаимодействия с операционной системой, в том числе и с файловой системой \cite{WASI}.
