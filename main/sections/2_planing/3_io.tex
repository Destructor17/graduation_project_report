Работа приложения подразумевает получение входных и выходных данных в том или ином виде.
WebAssembly предоставляет возможность среде выполнения определить функции, которые могут быть вызваны из приложения, работающего в этой среде.
Такие функции называются импортируемыми.
Импортируемые функции могут быть использованы для реализаций функций ввода и вывода данных.

В большинстве случаев память приложения с кодом WebAssembly представляет собой один непрерывный участок памяти, также называемый линейной памятью.
В этой памяти располагаются стек, куча, глобальные переменные и некоторые константы.

!!!Image there!!!

Тогда как многие библиотеки для выполнения кода WebAssembly предоставляют функции, позволяющие получить адрес начала линейной памяти и её размер, реализовать такой подход в среде выполнения кода, использующей WebAssembly API, невозможно, так как память основной части среды выполнения кода и память приложения находятся в двух разных объектах ArrayBuffer. 
Для реализации доступа к памяти приложения из среды выполнения кода используется подход, подобный подходу, использованному для вызова импортируемых функций из WebAssembly API.
Среда выполнения использует Atomics для уведомления Web Worker о том, что требуется произвести манипуляцию с памятью приложения, после чего Web Worker отправляет сообщение с объектом SharedArrayBuffer, содержащим запрошенным участком памяти, с которым среда выполнения кода может производить такие манипуляции, как чтение и запись данных.

Большинство приложений предоставляют пользовательский интерфейс, что делает их более интерактивными и простыми в использовании.
Сложность существующих программных библиотек для пользовательского интерфейса делает реализацию соответствующих импортируемых функций трудоёмкой.
В рамках данного дипломного проекта целесообразно использовать упрощённый подход: пользовательский интерфейс представляется в виде матрицы, элементы которой содержат информацию о символе, его цвете и цвете фона.
Для представления символа целесообразно использовать кодировку UTF-32, так как она поддерживает все существующие символы, и представление каждого символа требует одинакового количества байт.

Самым простым способом отобразить такую матрице является использование программной библиотеки NCurses.
NCurses позволяет работать с текстовыми пользовательскими интерфейсами на таких операционных системах, как Linux и MacOS.
Эмулятор терминала Windows сильно отличается от таковых у Linux и MacOS, потому NCurses их не поддерживает.
Для ОС Windows существует программная библиотека PDCurses, предоставляющая функционал, подобный таковому у NCurses.

На многих платформах, таких как Android, iOS и веб-браузеры, нет встроенного эмулятора терминала или его использование затруднительно.
Для таких платформ использование библиотек NCurses и PDCurses невозможно, поэтому для них пользовательский интерфейс реализован с помощью SDL.

Программная библиотека SDL предоставляет возможность простых операций с графикой, таких как отрисовка прямоугольников и копирование частей изображения.
Программная библиотека SDL\_ttf также предоставляет возможность отрисовки текста.
SDL и SDL\_ttf поддерживаются большинством существующих платформ.
На платформах Android, iOS и веб-браузерах для реализации пользовательского интерфейса используется только SDL и SDL\_ttf.
На других платформах существует возможность выбора. 
К примеру, для ОС Windows разрабатываемая среда выполнения кода предоставляет два исполняемых файла: webrogue.exe и webrogue\_sdl.exe.
webrogue.exe использует PDCurses, и открывает встроенный эмулятор терминала Windows, тогда как webrogue\_sdl.exe открывает новое окно, в котором пользовательский интерфейс реализован с помощью SDL.

В большинстве случаев приложения также манипулируют файлами.
WASI предоставляет набор стандартизированных импортируемых функций для взаимодействия с операционной системой, в том числе и с файловой системой \cite{WASI}.
WASI представлен в виде набора импортируемых функций.
Самой популярной реализацией WASI является программная библиотека UVWASI, использующая LibUV.
LibUV является кроссплатформенной программной библиотекой для доступа к функциям операционной системы.
LibUV создаёт ограничения к версиям операционной системы. 
К примеру, для ОС Android требуется версия Android 7.0 или выше.
Кроме того, при использовании компилятора Emscripten, для некоторых функций LibUV отсутствуют реализации.
Однако, так как эти функции не используются разрабатываемой средой выполнения кода, для них можно создать пустые реализации.
Так как при использовании метода нативного запекания программный код компилируется с помощью сторонних компиляторов, стандартные библиотеки которых не используют WASI, компиляция UVWASI и LibUV в данном случае не требуется.
