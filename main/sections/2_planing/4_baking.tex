Нативное запекание подразумевает компиляцию программного кода в машинный код напрямую, без использования WebAssembly.
Такой подход усложняет компиляцию программного кода, не позволяет запустить любое приложение, кроме того, для которого среда была скомпилирована, и не позволяет добавлять плагины после компиляции.
Использование такого подхода лишает среду выполнения многих преимуществ, но позволяет добиться наибольшей производительности и скорости запуска, а также значительно упрощает процесс отладки приложений.
К примеру, разработчик может установить точку останова как в программном коде своего приложения, так так и в коде среды выполнения, и запустить программу в любой интегрированной среде разработки, поддерживающей языки программирования C и C++.
Такой подход позволяет легко использовать функции подсветки синтаксиса, обнаружения ошибок и автодополнения кода в интегрированных средах разработки, не поддерживающих эти функции при компиляции в код WebAssembly, таких как Visual Studio или XCode.

Метод нативного запекания целесообразно использовать не только при разработке и отладке приложений, но для некоторых простых приложений, не требующих плагинов.
Одна из демонстраций разрабатываемой среды выполнения кода для веб-браузеров использует метод нативного запекания.
Для распространения более сложных приложений следует использовать компиляцию в код WebAssembly.

Wasm2c -- это программа, преобразующая код WebAssembly в код на языке программирования C.
При этом код WebAssembly может скомпилирован из программного кода других языков программирования, к примеру C++.

Запекание Wasm2c подразумевает компиляцию программного кода, полученного с помощью программы Wasm2c, в машинный код на этапе сборки среды выполнения кода.
В этом случае код WebAssembly выступает в роли промежуточного представления.
Такой подход, как и метод нативного запекания, обеспечивает высокую производительности и скорость запуска приложения, но не позволяет запустить любое приложение, кроме того, для которого среда была скомпилирована, и не позволяет добавлять плагины после компиляции.

Главным преимуществом данного подхода перед методом нативного запекания является простота компиляции.
При использовании метода запекания Wasm2c разработчику не нужно волноваться о том, чтобы его программный код мог быть скомпилирован под все поддерживаемые платформы.
К примеру, если разработчик, использующий ОС Linux, создаёт приложение для разрабатываемой среды выполнения кода, он может быть уверен, что с помощью метода запекания Wasm2c это приложение может быть собрано для других ОС, таких как Windows или Android.
