В рамках данного дипломного проекта части кода, которые выступают в качестве приложений, библиотек и плагинов называются <<модами>>. 
Слово <<мод>> было выбрано как общая часть слов <<модуль>> и <<модификация>>.

Стандартные библиотеки C и C++, а также некоторые другие функции вынесены в отдельный мод, называемый виртуальным ядром, или просто ядром.
Данный мод встроен в среду выполнения кода, что избавляет от необходимости многократно включать его в приложения.

Для распространения модов следует использовать удобный для пользователя формат.
В рамках дипломного проекта разработан формат, в котором приложения представлены в виде одного файла-архива.
Такие файлы имеют расширение wrmod.
Файл имеет две части - заголовочник и тело.
В заголовочнике находится информация, требующая упрощенного доступа, такая как название мода.
Телом файла является архив.
Для уменьшения размера архива может быть применено сжатие XZ или Zstandart.
Сжатие не является обязательным.
Сжатие не применимо для заголовочника.
Архив содержащий статичную библиотеку с исполняемым кодом и другие файлы, называемые ресурсами.
Статичная библиотека используется в процессе линковки для получения кода запускаемого приложения, тогда как само приложение может получить доступ к ресурсам с помощью импортируемых функций.
