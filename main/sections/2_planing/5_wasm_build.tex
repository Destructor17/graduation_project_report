В рамках данного дипломного проекта качестве языка программирования для примеров приложений, запускаемых в среде выполнения, выбран C++.
Для компиляции программного кода на языке программирования С++ в код WebAssembly существует два компилятора: Emscripten и Clang. 
Emscripten предназначен для компиляции приложений, выполняемых в веб-браузерах, сильно зависит от кода на языке программирования JavaScript и не позволяет использовать стандартные библиотеки языков программирования C и С++, реализованных с помощью WASI.
Clang лишен перечисленных недостатков, поэтому его использование более целесообразно.
В качестве системы сборки приложений будет использоваться GNU Make, конфигурации которого сгенерированы с помощью CMake.
Для ОС Windows, где обычно установка GNU Make затруднительна, используется NMake, который может быть установлен вместе с Visual Studio.
В качестве стандартных библиотек языков программирования C и C++ используется реализации этих библиотек от WASI-SDK.
