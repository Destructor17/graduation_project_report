При создании кросскомпилированных приложений наибольшую гибкость показывают подходы, полагающиеся на промежуточное представление кода, и использующие интерпретацию, JIT-компиляцию или компиляцию в нативный код в различных ситуациях.

В процессе разработки важную часть играют такие инструменты, являющиеся частью интегрированной среды разработки, как отладчик.
Многие популярные интегрированной среды разработки, такие как Visual Studio и XCode, имеют встроенные отладчики, которые удобно использовать для отладки нативного кода, если использован компилятор, предоставленный соответствующей средой разработки.

Учитывая требования к проектируемой среде выполнения кода, оптимальным решением будет использование WebAssembly в качестве байт-кода и промежуточного представления с сохранением возможности компиляции программного кода в нативный с использованием сторонних компиляторов.
