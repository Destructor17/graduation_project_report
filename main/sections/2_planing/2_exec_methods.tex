При создании кросскомпилированных приложений наибольшую гибкость показывают подходы, полагающиеся на промежуточное представление кода, и использующие интерпретацию, JIT-компиляцию или компиляцию в нативный код в различных ситуациях.

При этом, если разработчик решит использовать для создания приложений такие языки программирования, как C или C++, возможность использовать в процессе разработки такие инструменты, являющиеся частью интегрированной среды разработки, как отладчик, значительно упростит разработку. 

Учитывая требования к проектируемой среде выполнения кода, оптимальным решением будет использование WebAssembly с сохранением возможности компилляции в нативный код с использованием сторонних компиляторов.

% Так как функция динамической линковки кода WebAssembly является экспериментальной \cite{WASMDynamicLinking}, для объединения множества частей кода в одну следует использовать статичную линковку.