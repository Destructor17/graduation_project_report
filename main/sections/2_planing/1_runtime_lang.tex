Выбор языка программирования зависит от назначения программного обеспечения.

Для среды выполнения кода наиболее подходящим языком программирования является C++.
C++ -- это мощный, статически типизированный язык программирования общего назначения.
Большая часть программных библиотек, связанных со средами выполнения кода и компиляторами, написаны на языках C и C++.
C++ сохраняет большую часть синтаксиса и семантики языка C, поэтому программа, написанная на C++, может легко использовать код, написанный на C.
Кроме того, такие возможности языка C++, как объектно-ориентированное программирование и шаблоны, значительно упрощают разработку по сравнению с использованием языка C.

Процесс компиляции кода на языке программирования С++ для различных платформ может сильно различаться.
К примеру, для ОС Windows обычно используются средства сборки интегрированной среды разработки Visual Studio, а для MacOS -- XCode.
Конфигурации проектов Visual Studio и XCode определяются в различных, несовместимых друг с другом форматах.
По этой причине, если в структуру проекта будут внесены такие изменения, как добавления новых файлов с исходным кодом или программных библиотек, эти изменения необходимо будет отразить в файлах конфигурации для всех платформ, что увеличивает трудоёмкость разработки.

CMake -- это кроссплатформенная система автоматизации сборки программного обеспечения. Ее основное назначение - упростить процесс сборки приложений для различных платформ. 
CMake предназначен для работы с различными языками программирования, в том числе и C++.
CMake может генерировать файлы проектов для различных интегрированных сред разработки, таких как Visual Studio и Xcode, что облегчает разработку и отладку приложений в удобной среде разработки.
Android SDK имеет встроенную поддержку CMake.

При создании кросскомпилированных приложений наибольшую гибкость показывают подходы, полагающиеся на промежуточное представление кода, и использующие интерпретацию, JIT-компиляцию или компиляцию в нативный код в различных ситуациях.

В процессе разработки важную часть играют такие инструменты, являющиеся частью интегрированной среды разработки, как отладчик.
Многие популярные интегрированной среды разработки, такие как Visual Studio и XCode, имеют встроенные отладчики, которые удобно использовать для отладки нативного кода, если использован компилятор, предоставленный соответствующей средой разработки.

Учитывая требования к проектируемой среде выполнения кода, оптимальным решением будет использование WebAssembly в качестве байт-кода и промежуточного представления с сохранением возможности компиляции программного кода в нативный с использованием сторонних компиляторов.

Назначение приложений, запускаемых в проектируемой среде выполнения, может быть произвольным. 
Кроме того, многие разработчики склонны выбирать язык программирования на основании их навыков.
По этим причинам для различных приложений может понадобиться использовать различные языки.
Область применения проектируемой среды выполнения будет гораздо шире, если данная среда будет иметь возможность создавать приложения на нескольких языках программирования, а так же возможность легко добавлять новые языки программирования.
По этой причине WebAssembly поддерживает большое количество языков программирования \cite{WASMLanguages}.
Язык программирования поддерживает WebAssembly, если выполняется как минимум одно из условий:
\begin{itemize}
    \item[-] исходный код на этом языке может быть скомпилирован в WebAssembly;
    \item[-] исходный код на этом языке может быть преобразован в код другого языка программирования, который поддерживает WebAssembly;
    \item[-] для этого язык программирования существует интерпретатор, написанный на языке программирования, который поддерживает WebAssembly.
\end{itemize}

Так как C++ выбран в качестве основного языка программирования в данном проекте, а также из-за популярности данного языка программирования, его использование в качестве языка примеров кода приложений, запускаемых в создаваемой среде выполнения, является целесообразным.
