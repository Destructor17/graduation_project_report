Выбор языка программирования в зависит от назначения программного обеспечения.

Для среды выполнения кода наиболее подходящим языком программирования является C++.
Большая часть программных библиотек, связанных со средами выполнения кода и компилляторами, написаны на языках C и C++.
C++ позволяет легко использовать код, написанный на C.
Кроме того, такие возможности языка C++, как объектно-ориентированное программирование и шаблоны, значительно упрощают разработку по сравнению с использованием языка C.

Процесс компилляции кода на языке программирования С++ для различных платформ может сильно различаться.
К примеру, для ОС Windows обычно используются средства сборки интегрированной среды разработки Visual Studio, а для MacOS -- XCode.
Конфигурации проектов Visual Studio и XCode определяются в различных, несовместимых друг с другом форматах.
По этой причине, если в структуру проекта будут внесены такие изменения, как добавления новых файлов с исходным кодом или программных библиотек, эти изменения необходимо будет отразить в файлах конфигурации для всех платформ, что увеличивает трудоёмкость разработки.

CMake -- это кроссплатформенная система автоматизации сборки программного обеспечения. Ее основное назначение - упростить процесс сборки приложений для различных платформ. 
CMake предназначен для работы с различными языками программирования, в том числе и C++.
CMake может генерировать файлы проектов для различных интегрированных сред разработки, таких как Visual Studio и Xcode, что облегчает разработку и отладку приложений в удобной среде разработки.
Android SDK имеет встроенную поддержку CMake.

