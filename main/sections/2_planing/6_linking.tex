Разрабатываемая среда выполнения кода должна поддерживать возможность добавления плагинов к программам.
Запуск основной программы и плагинов в изолированных друг от друга средах позволит добиться поставленной цели, но сильно усложнит организацию взаимодействия приложения и плагинов.
Альтернативным подходом является компоновка приложения и плагинов перед их выполнением.
Компоновка подразумевает объединение множества частей кода в одну.
Компоновка позволит запустить приложение и плагины без изоляции друг между другом, что значительно упростит разработку подобных приложений.

Существует два метода компоновки: динамический и статичный.
Динамическая компоновка обычно производится прямо перед выполнением программы.
В ОС Windows файлы библиотек, предназначенные для динамической компоновки, имеют расширение <<.dll>>, а в Linux -- <<.so>>.
Статичная компоновка производится во время сборки приложений.
В ОС Windows файлы библиотек, предназначенные для статичной компоновки, имеют расширение <<.lib>>, а в Linux -- <<.a>>.
Так как функция динамической компоновка кода WebAssembly является экспериментальной \cite{WASMDynamicLinking}, в разрабатываемой среде выполнения кода следует использовать статичную компоновка.

Для статичной компоновки кода WebAssembly обычно используется утилита LLD, которая основана на программной библиотеке LLVM.
Несмотря на то, что LLD является кроссплатформенной утилитой, её использование не целесообразно из-за сложности компиляции библиотеки LLVM.
Репозиторий LLVM имеет размер около 2.5 гигабайт, а процесс её компиляции занимает около часа на 8-и ядерном компьютере и требует порядка 1 гигабайта оперативной памяти на каждое ядро.
Процесс настройки конфигураций CMake для компиляции LLVM для различных платформ оказался крайне затруднительным, и использование утилиты LLD в качестве библиотеки увеличивает размер исполняемого файла разрабатываемой среды выполнения кода примерно на 20 MБ.

По этим причинам в рамках дипломного проекта разработана программная библиотека compact\_linker. 
Эта библиотека частично повторяет функционал утилиты LLD, однако, в отличие от последней, намного проще в компиляции и использовании.
Код этой библиотеки включает в себя сильно видоизмененные части кода проекта WABT, используемые для работы с кодом WebAssembly.
