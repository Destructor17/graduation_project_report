Для демонстрации возможности интерактивного взаимодействия с приложением, запускаемым в разрабатываемой среде выполнения кода, в рамках дипломного проекта создано приложение log2048.
Данное приложение является игрой-пазлом, где в поле 4 на 4 клетки располагаются натуральные числа или пустые клетки.
Пользователь может перемещать числа в одном из четырёх направлений с помощью клавиатуры, мыши или сенсорного экрана.
При этом все числа перемещаются в выбранном направлении до тех пор, пока не столкнутся с краем поля или другим числом. 
Если сталкиваются два одинаковых числа, то они превращаются в одно число, на единицу большее, чам исходные.
При каждом перемещении случайная свободная клетка превращается в число 1.
Игра продолжается до тех пор, пока есть свободные клетки.

Приложение log2048 использует СУБД SQLite для сохранения прогресса пользователя между запусками приложения.
SQLite может быть скомпилирован в WebAssembly и получать доступ к файловой системе с помощью WASI, но для демонстрации возможностей импортируемых функций SQLite подключен к самой среде выполнения кода, а приложение получает доступ к СУБД с помощью импортируемых функций.
Для более удобной работы с СУБД приложение log2048 использует ORM-библиотеку Sqlpp11.

Для простоты тестирования разрабатываемой среде выполнения кода мод log2048 распространяется вместе с самой средой.

В качестве примера приложения использующего файлы-ресурсы, в рамках данного дипломного проекта создано приложение под названием rugpt2.
Данное приложение использует нейронную модель rugpt3-small-based-on-gpt2, созданную SberDevices, для генерации текста.

Приложение rugpt2 использует программную библиотеку для машинного обучения GGML.
Для повышения производительности таких операций, как матричное произведение, библиотека GGML использует WASM-SIMD.
Все программные библиотеки, предназначенные для выполнения кода WebAssembly, и используемые в разрабатываемой среде, поддерживают WASM-SIMD.
Исключением является Wasm3, которая используется только на iOS и только в исключительной ситуации -- если предустановленная библиотека JavaScriptCore не поддерживает WebAssembly.

Для того, чтобы продемонстрировать модульность, а также возможность использовать различные языки программирования, в рамках дипломного проекта разработано простое приложение и несколько плагинов для него.

Приложение имеет название langExampleCore.
При запуске это приложение отображает вертикальный список строк.
Плагины добавляют элементы в этот список. 
В случае, если не подключены плагины, отображается только встроенный элемент.
