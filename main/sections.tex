\GPRSection{АНАЛИЗ ПРЕДМЕТНОЙ ОБЛАСТИ}{1_analiz}
\GPRSection{ПРОЕКТИРОВАНИЕ СИСТЕМЫ}{2_planing}
\GPRSection{РЕАЛИЗАЦИЯ И ИСПЫТАНИЕ СИСТЕМЫ}{3_implementation}
% \GPRSectionFile{ЭНЕРГОСБЕРЕЖЕНИЕ}{6_energy_saving}
\GPREconomicSection{
    Любая разработка программного обеспечения, приложения или их модулей требует финансовых и человеческих ресурсов.

    Задачей дипломного проекта является разработка среды выполнения кода используя язык программирования C++.

    Разработка элементов среды выполнения кода предусматривает
    проведение практически всех стадий проектирования и относится ко второй группе сложности.
    
    Последовательность расчетов:
    \begin{itemize}
        \item[1.] Расчёт объёма функций программных модулей.
        \item[2.] Расчёт полной себестоимости.
        \item[3.] Расчёт цены и прибыли по программному продукту.
    \end{itemize}
}{
    На разработку подобных решений, таких как Java Virtual Machine, .NET и скриптовых движков, уходит от сотен тысяч до сотен миллионов долларов в зависимости от сложности проекта. 
    Среда выполнения кода, разработанная в рамках дипломного проектирования, отличается от имеющихся решений не только большей гибкостью, но и значительно меньшими затратами на разработку, что связано с использованием готовых решений, таких как компилляторы, системы сборки, интерпретаторы JIT-компилляторы и другие.
}
