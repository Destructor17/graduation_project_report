В ходе разработки данного дипломного проекта была создана среда выполнения кода под названием Webrogue, позволяющая запускать код на различных платформах.
Данная среда имеет ряд преимуществ перед существующими решениями.

В отличии от использования обычных кросскомпилляторов кода на языках C и C++, Webrogue не требует адаптации кода для поддерживаемых платформ. 
Webrogue следует принципу "работает на одной платформе - работает везде".

В отличии от большинства виртуальных машин, Webrogue прост в установке и позволяет легко использовать уже существующие библиотеки на языках C и C++.

Webrogue использует интерпретатор лишь в исключительных случаях, что позволяет добиться высокой производительности при выполнении кода.

Webrogue позволяет легко создавать плагины для приложений.

Метод нативного запекания позволяет использовать популярные IDE, такие как Microsoft Visual Studio, Visual Studio Code, XCode и Android Studio, для разработки приложений, а так же для поиска и устранения ошибок.

В рамках дипломного проекта были разработаны:
\begin{itemize}
    \item[-] приложение для Windows;
    \item[-] установщики приложения для Windows в виде программы и в виде пакета MSIX;
    \item[-] приложение для Linux в формате AppImage;
    \item[-] пакеты для менеджеров пакетов DEB и RPM;
    \item[-] приложение для MacOS;
    \item[-] приложение для Android;
    \item[-] приложение для iOS;
    \item[-] веб-сайт;
    \item[-] 4 интерактивных веб-приложения;
    \item[-] автоматическая сборка всех разработанных приложений в облаке.
\end{itemize}

В процессе разработки были использованы:
\begin{itemize}
    \item[-] программное обеспечение CMake;
    \item[-] системы сборки GNU Makefiles, Ninja-build, MSBuild, XCBuild и NMake Makefiles;
    \item[-] компиляторы GCC, Clang, MSVC, Emscripten и Apple-Clang;
    \item[-] набор инструментов WASI-SDK;
    \item[-] программные библиотеки для интерпретации и JIT-компиляции кода Wasmer, Wasmtime, WAMR и WASM3;
    \item[-] библиотеки для взаимодействия с пользователем ncurses, PDCurses и SDL;\item[-] библиотека СУБД SQLite;
    \item[-] библиотека для машинного обучения GGML;
    \item[-] библиотеки для чтения сжатых данных XZ и Zstd;
    \item[-] библиотека compactlinker, также разработанная в рамках дипломного проекта.
\end{itemize}

Среда выполнения кода Webrogue упрощает создание кроссплатформенных приложений, что и является основной целью дипломного проекта.
