В ходе разработки данного дипломного проекта была создана среда выполнения кода (runtime), позволяющая запускать приложения на различных платформах.
Данная среда имеет ряд преимуществ перед существующими методами создания кроссплатформенных приложений.

В отличии от использования обычных компиляторов кода на языках C и C++, разработанная среда выполнения не требует адаптации кода для различных платформ. 

В отличии от большинства виртуальных машин, разработанная среда проста в установке и позволяет легко использовать уже существующие библиотеки на языках программирования C и C++.

Созданная среда выполнения кода использует интерпретатор лишь в исключительных случаях, отдавая предпочтение JIT-компиляции, что позволяет добиться высокой производительности при выполнении кода.

Высокая модульность позволяет легко создавать плагины для приложений.

Метод нативного запекания позволяет использовать популярные IDE, такие как Microsoft Visual Studio, Visual Studio Code, XCode и Android Studio для разработки приложений, работающих в созданной среды выполнения кода, а так же для поиска и устранения ошибок.

В рамках дипломного проекта были разработаны:

-- ~реализация среды выполнения для Windows, а также её установщики в виде программы и в виде пакета MSIX;

-- ~реализация среды выполнения для Linux в формате AppImage и пакеты для менеджеров пакетов DEB и RPM;

-- ~реализация среды выполнения для Android;

-- ~реализации среды выполнения для MacOS и iOS;

-- ~веб-сайт с кратким описанием проекта, ссылками для скачивания и примером реализации среды выполнения, работающим в браузере и не требующим установки;

-- ~примеры приложений, запускаемых в среде Webrogue;

-- ~конфигурация Github Actions для автоматической сборки всего разработанного программного обеспечения в облаке.

Разработанная среда выполнения кода упрощает создание кроссплатформенных приложений, что и является основной целью дипломного проекта.
