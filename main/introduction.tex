Существует множество операционных систем и платформ: Linux, Windows, MacOS, Android, iOS, веб-браузеры и другие.
Многие из них тесно связаны друг с другом, однако, в общем случае, приложение, предназначенное лишь для одной платформы, невозможно или проблематично запустить на другой.
К примеру, если приложение создано для платформы iOS, то, чтобы пользоваться таким приложением, пользователю прийдется купить соответствующее устройство от Apple, которое может быть дорогостоящим.

Самым очевидным решением создание отдельных, независимых приложений для различных платформ.
Однако, главным недостатком такого подхода является увеличение затрат на разработку и поддержку приложения. 
К примеру, для того, чтобы создать независимые приложения для платформ Windows и Android, потребуется две команды разработчиков.

Более экономичным решением является использование общей части программного кода на нескольких платформах.
На сегодняшний день существует множество решений, позволяющих создавать подобные приложения, но каждое из них имеют как минимум один из недостатков:
\begin{itemize}
    \item[-] сложность сборки приложения;
    \item[-] невозможность использования популярных программных библиотек;
    \item[-] необходимость использования специальных языков программирования;
    \item[-] невозможность расширения функционала с помощью плагинов;
    \item[-] низкая производительность.
\end{itemize}

Целью дипломного проектирования является создание кроссплатформенной среды выполнения кода, объединяющей преимущества существующих подходов к созданию кроссплатформенных приложений, что позволит упростить и, следовательно, ускорить и удешевить их разработку таких приложений.
