В контексте разработки приложений существует множество целевых платформ: Linux, Windows, MacOS, Android, iOS, веб-браузеры и другие.
Многие из них тесно связаны друг с другом, однако, в общем случае, приложение, предназначенное лишь для одной платформы, невозможно или проблематично запустить на другой.
К примеру, если приложение создано для iOS, то, чтобы пользоваться таким приложением, пользователю прийдется купить соответствующее устройство от Apple, которое может быть дорогостоящим.

Создание отдельных приложений с нуля для каждой целевых платформы увеличивает затраты на разработку и поддержку приложения. 
К примеру, для того, чтобы создать независимые приложения для ОС Windows и Android, потребуется две команды разработчиков.
Более экономичным решением является использование общей части программного кода на нескольких платформах.
На сегодняшний день существует множество решений, позволяющих создавать подобные приложения, и каждое из них обладают некоторыми из следующих преимуществ, но не всеми сразу:
\begin{itemize}
    \item[-] поддержка большинства популярных платформ;
    \item[-] отсутствие роста сложности разработки приложения с увеличением количества поддерживаемых платформ;
    \item[-] отсутствие необходимости использования специальных языков программирования;
    \item[-] возможность использования популярных программных библиотек;
    \item[-] возможность расширения функционала с помощью плагинов;
    \item[-] высокая производительность.
\end{itemize}

Целью дипломного проектирования является создание кроссплатформенной среды выполнения кода, объединяющей преимущества существующих подходов к созданию кроссплатформенных приложений, что позволит упростить и, следовательно, ускорить и удешевить разработку таких приложений.
